%% Generated by Sphinx.
\def\sphinxdocclass{report}
\documentclass[letterpaper,10pt,ngerman]{sphinxmanual}
\ifdefined\pdfpxdimen
   \let\sphinxpxdimen\pdfpxdimen\else\newdimen\sphinxpxdimen
\fi \sphinxpxdimen=.75bp\relax

\PassOptionsToPackage{warn}{textcomp}
\usepackage[utf8]{inputenc}
\ifdefined\DeclareUnicodeCharacter
% support both utf8 and utf8x syntaxes
  \ifdefined\DeclareUnicodeCharacterAsOptional
    \def\sphinxDUC#1{\DeclareUnicodeCharacter{"#1}}
  \else
    \let\sphinxDUC\DeclareUnicodeCharacter
  \fi
  \sphinxDUC{00A0}{\nobreakspace}
  \sphinxDUC{2500}{\sphinxunichar{2500}}
  \sphinxDUC{2502}{\sphinxunichar{2502}}
  \sphinxDUC{2514}{\sphinxunichar{2514}}
  \sphinxDUC{251C}{\sphinxunichar{251C}}
  \sphinxDUC{2572}{\textbackslash}
\fi
\usepackage{cmap}
\usepackage[T1]{fontenc}
\usepackage{amsmath,amssymb,amstext}
\usepackage{babel}



\usepackage{times}
\expandafter\ifx\csname T@LGR\endcsname\relax
\else
% LGR was declared as font encoding
  \substitutefont{LGR}{\rmdefault}{cmr}
  \substitutefont{LGR}{\sfdefault}{cmss}
  \substitutefont{LGR}{\ttdefault}{cmtt}
\fi
\expandafter\ifx\csname T@X2\endcsname\relax
  \expandafter\ifx\csname T@T2A\endcsname\relax
  \else
  % T2A was declared as font encoding
    \substitutefont{T2A}{\rmdefault}{cmr}
    \substitutefont{T2A}{\sfdefault}{cmss}
    \substitutefont{T2A}{\ttdefault}{cmtt}
  \fi
\else
% X2 was declared as font encoding
  \substitutefont{X2}{\rmdefault}{cmr}
  \substitutefont{X2}{\sfdefault}{cmss}
  \substitutefont{X2}{\ttdefault}{cmtt}
\fi


\usepackage[Sonny]{fncychap}
\ChNameVar{\Large\normalfont\sffamily}
\ChTitleVar{\Large\normalfont\sffamily}
\usepackage{sphinx}

\fvset{fontsize=\small}
\usepackage{geometry}


% Include hyperref last.
\usepackage{hyperref}
% Fix anchor placement for figures with captions.
\usepackage{hypcap}% it must be loaded after hyperref.
% Set up styles of URL: it should be placed after hyperref.
\urlstyle{same}

\addto\captionsngerman{\renewcommand{\contentsname}{Inhalt}}

\usepackage{sphinxmessages}
\setcounter{tocdepth}{1}



\title{Shirtnetwork}
\date{25.06.2020}
\release{}
\author{Alexander Kludt}
\newcommand{\sphinxlogo}{\vbox{}}
\renewcommand{\releasename}{}
\makeindex
\begin{document}

\ifdefined\shorthandoff
  \ifnum\catcode`\=\string=\active\shorthandoff{=}\fi
  \ifnum\catcode`\"=\active\shorthandoff{"}\fi
\fi

\pagestyle{empty}
\sphinxmaketitle
\pagestyle{plain}
\sphinxtableofcontents
\pagestyle{normal}
\phantomsection\label{\detokenize{index::doc}}


Mit Shirtnetwork haben Sie sich für eine einzigartige und umfassende Lösung für Print on Demand im Webshop entschlossen.
Mit einer einzigen Software können Sie nicht nur individualisierte Artikel direkt vom Kunden gestalten lassen,
sondern auch durch den viralen Effekt von Partnershops und dem Marktplatz Ihre Umsätze erheblich steigern.

Durch unser einfaches und leicht durchschaubares Modell haben Sie die Kosten jederzeit im Griff,
denn es fallen nur dann Gebühren an wenn Sie auch wirklich einen Artikel verkaufen.

Wir helfen Ihnen jederzeit kompetent bei allen Fragen und Problemen, denn Ihr Geschäftserfolg ist ein unser wichtigstes Interesse.

Der Einstieg ist kinderleicht und kann von Ihren in\sphinxhyphen{}house Kräften oder von einer Medienagentur Ihrer Wahl vor\sphinxhyphen{}genommen werden. N
atürlich können Sie auch unsere verschiedenen Services in Anspruch nehmen um schnellst möglich beginnen zu können.
Von der Grundinstallation bis zur kompletten Layout und Konzept Entwicklung können wir Ihnen alle Dienstleistungen anbieten.

Durch unsere modulare Architektur schreiben wir Ihnen die technische Grundlage Ihres Webshops nicht vor,
entscheiden Sie sich selbst welches Shop System Ihren Ansprüchen am besten gerecht wird und installieren Sie einfach eines unserer zahlreichen Module für das entsprechende System.

Die Datenpflege der Produkte wird in unserem speziellen auf HTML5 basierenden Webanwendung vorgenommen, dadurch werden Ihnen lange Wartezeiten erspart.

90\% der Modul Quellcodes sind offen und können nach belieben angepasst werden, dadurch können auch die speziellsten Anwendungsfälle abgebildet werden.
Fangen Sie schon heute damit an Ihre Werbeartikel mit Shirtnetwork zu vertreiben und erreichen Sie eine nie da gewesenen Absatzsteigerung durch den Einsatz von der interaktiven Technik die wir Ihnen bieten können.


\chapter{Erste Schritte}
\label{\detokenize{intro/index:erste-schritte}}\label{\detokenize{intro/index::doc}}
Um mit Shirtnetwork arbeiten zu können benötigen Sie einen Shirtnetwork Benutzer.
Anschließend können Sie sich im sog. HBE (Händler Backend) anmelden und mit der Datenpflege beginnen.


\section{Anmeldung}
\label{\detokenize{intro/register:anmeldung}}\label{\detokenize{intro/register::doc}}
Füllen Sie alle Pflichtfelder in unserem \sphinxhref{https://www.shirtnetwork.de/konto-eroeffnen}{Registrierungsformular} aus, die Kontoerstellung ist kostenlos.
Die gewählte E\sphinxhyphen{}Mail Adresse wird auch gleichzeitig als ihr zukünftiger Shirtnetwork Benutzername verwendet.

Sie erhalten automatisch 5 Coins um Testbestellungen auszuführen.
Weitere Informationen zu unseren Preisen und den Zahlungsmodalitäten entnehmen Sie bitte dem entsprechenden Handbucheintrag.


\section{Kontoübersicht}
\label{\detokenize{intro/account:kontoubersicht}}\label{\detokenize{intro/account::doc}}
Nachdem Sie einen Benutzer erstellt haben erlangen Sie Zugriff auf Ihren \sphinxhref{https://www.shirtnetwork.de/mein-konto}{Kontobereich}. Hier können Sie Ihre Stammdaten
ändern, den Newsletter abonnieren, Ihre Gebührenübersicht aufrufen sowie Guthaben aufladen.


\subsection{Verkaufsübersicht}
\label{\detokenize{intro/account:verkaufsubersicht}}
Hier finden Sie einen Überblick über alle von uns registrierten Verkäufe und die erhobenen Gebühren.
Geben Sie den gewünschten Datumsbereich in die entsprechenden Felder ein und klicken Sie auf \sphinxguilabel{absenden}.


\subsubsection{Verkäufe stornieren}
\label{\detokenize{intro/account:verkaufe-stornieren}}
Sollte eine Bestellung vom Kunden oder von Ihnen storniert worden sein haben Sie die Möglichkeit sich
die Gebühren hierfür gutschreiben zu lassen. Wenn Sie Verkäufe stornieren möchten wählen Sie die jeweilige Zeile aus und geben Sie den Stornierungsgrund an,
klicken Sie auf \sphinxguilabel{absenden} um die Stornierung durchzuführen.

\begin{sphinxadmonition}{attention}{Achtung:}
Wir behalten uns vor Stichproben anzufordern und bei unbegründeten Stornierungen die jeweiligen Forderungen nachzuberechnen und den Service ohne Vorankündigung zu kündigen.
\end{sphinxadmonition}


\subsection{Guthaben aufladen}
\label{\detokenize{intro/account:guthaben-aufladen}}
Kunden die nicht mit dem Lastschrift Verfahren arbeiten müssen Guthaben (Coins) auf Ihr Kundenkonto aufladen um
personalisierte Artikel über Shirtnetwork zu verkaufen. Wählen Sie den Punkt \sphinxhref{https://www.shirtnetwork.de/shop/shirtnetwork-coins-aufladen}{Coins aufladen} und folgen Sie den Schritten
um den gewünschten Betrag aufzuladen.


\section{Produktdaten hinzufügen}
\label{\detokenize{intro/hbe:produktdaten-hinzufugen}}\label{\detokenize{intro/hbe::doc}}
Da Shirtnetwork Artikel mehr Daten als ein klassischer Artikel benötigen wie z.B. verschiedene Ansichten, Druckarten etc.
ist es notwendig die Artikeldaten in einer separaten Software anzulegen. Die Daten können massenweise importiert
oder manuell eingegeben werden, im Anschluss können die notwendigen Daten in Ihr Shopsystem übertragen werden.

Die Verwaltungssoftware HBE finden Sie online, sie lässt sich in den meisten modernen Browsern bedienen und ist Endgeräte unabhängig.
Sie finden das HBE unter \sphinxurl{https://hbe.shirtnetwork.de}

Hinweise zur Bedienung der Software entnehmen Sie bitte dem zugehörigen \sphinxhref{/hbe/index}{Handbucheintrag}.


\section{Backend Server installieren}
\label{\detokenize{intro/backend:backend-server-installieren}}\label{\detokenize{intro/backend::doc}}
Alle vom Kunden bereit gestellten Daten wie hochgeladene Bilder oder die gestalteten Produkte werden in einem zusätzlichen
System abgelegt. Dieser sog. Backend Server kann von beliebig vielen Ihrer Systeme genutzt werden und muss nur einmalig
installiert werden.

Folgen Sie bitte der {\hyperref[\detokenize{tools/backend::doc}]{\sphinxcrossref{\DUrole{doc}{Anleitung zur Installation des Backend Servers}}}}


\section{Shop verbinden}
\label{\detokenize{intro/connect:shop-verbinden}}\label{\detokenize{intro/connect::doc}}
Damit der Designer in Ihrem Shop dargestellt werden kann ist ein sogenannter Connector notwendig. Hierbei handelt es sich
um ein Shop Plugin das wie gewohnt installiert wird.

Bitte rufen Sie den zugehörigen Handbucheintrag für Ihr Shopsystem auf und installieren Sie den Connector:


\subsection{Shop Module}
\label{\detokenize{modules/index:shop-module}}\label{\detokenize{modules/index::doc}}
Damit der Designer in Ihrem Shop dargestellt werden kann ist ein sogenannter Connector notwendig. Hierbei handelt es sich
um ein Shop Plugin das wie gewohnt installiert wird.


\subsubsection{Allgemeine Informationen}
\label{\detokenize{modules/general/index:allgemeine-informationen}}\label{\detokenize{modules/general/index::doc}}

\paragraph{Artikelnummern Schema}
\label{\detokenize{modules/general/skuscheme:artikelnummern-schema}}\label{\detokenize{modules/general/skuscheme::doc}}
Die Artikelnummer ist eine zentrale Komponente die sehr wichtig ist für die Übertragung der Daten an das Shop System ist.
Die verschiedenen Connectoren unterstützen unterschiedliche Methoden der Übertragung.
Die folgenden Schemas sind lediglich Empfehlungen die sich in der Praxis bewährt haben,
das soll Sie natürlich nicht davon abhalten Ihr eigenes Schema zu verwenden.


\subparagraph{Stammartikel}
\label{\detokenize{modules/general/skuscheme:stammartikel}}
Für alle Connectoren gilt, die Artikelnummer des Stammartikels muss im HBE und im Shopsystem genau identisch sein.
Wir empfehlen entweder die Übernahme der Artikelnummer des Händlers oder
die Verwendung eines Schemas um Artikelnummern festzulegen.


\subparagraph{Nummeriert}
\label{\detokenize{modules/general/skuscheme:nummeriert}}
Eine einfache Möglichkeit der Artikelnummernvergabe ist die Verwendung von
fortlaufenden Nummern mit führenden Nullen.

\begin{sphinxVerbatim}[commandchars=\\\{\}]
\PYG{l+m+mi}{000001}
\PYG{l+m+mi}{000002}
\PYG{l+m+mi}{000003}
\PYG{l+m+mi}{000004}
\PYG{l+m+mi}{000005}
\end{sphinxVerbatim}


\subparagraph{Gruppiert}
\label{\detokenize{modules/general/skuscheme:gruppiert}}
Wenn Sie viele Warengruppen im Sortiment haben empfiehlt es sich die Artikel zusätzlich mit einem Prefix zu versehen
um z.B. zwischen Herren, Damen und Kinderkleidung zu unterscheiden.

\begin{sphinxVerbatim}[commandchars=\\\{\}]
\PYG{n}{A00001}
\PYG{n}{A00002}
\PYG{n}{B00001}
\PYG{n}{B00002}
\PYG{n}{C00001}
\PYG{n}{C00002}
\end{sphinxVerbatim}

\begin{sphinxadmonition}{hint}{Hinweis:}
Diese Schemas lassen sich auch für die Varianten wiederverwenden.
\end{sphinxadmonition}


\subparagraph{Varianten}
\label{\detokenize{modules/general/skuscheme:varianten}}
Wie Sie die Varianten Ihrer Artikel kennzeichnen sollten hängt davon ab ob Sie eine Lagerhaltung
und/oder eine Anbindung an Ihre Warenwirtschaft benötigen. 2 Methoden haben sich bisher gut bewährt.


\subparagraph{Sub Nummern}
\label{\detokenize{modules/general/skuscheme:sub-nummern}}
Bei diesem Schema werden die Varianten jeweils mit der Hauptartikelnummer als Prefix gekennzeichnet.
Beispiel für Varianten eines Hauptartikels mit der Artikelnummer 000001:

\begin{sphinxVerbatim}[commandchars=\\\{\}]
\PYG{l+m+mi}{000001}\PYG{o}{\PYGZhy{}}\PYG{l+m+mi}{001}
\PYG{l+m+mi}{000001}\PYG{o}{\PYGZhy{}}\PYG{l+m+mi}{002}
\PYG{l+m+mi}{000001}\PYG{o}{\PYGZhy{}}\PYG{l+m+mi}{003}
\PYG{l+m+mi}{000001}\PYG{o}{\PYGZhy{}}\PYG{l+m+mi}{004}
\PYG{l+m+mi}{000001}\PYG{o}{\PYGZhy{}}\PYG{l+m+mi}{005}
\end{sphinxVerbatim}


\subparagraph{Variantenabhängig}
\label{\detokenize{modules/general/skuscheme:variantenabhangig}}
Diese Methode empfiehlt sich um verschiedene Varianten unterschiedlicher Artikel zu gruppieren \sphinxhyphen{}
z.B. nach Farbe oder Form. Hierzu vergeben Sie doppelte Variantenartikelnummern innerhalb unterschiedlicher Artikel.

Varianten für Artikel Männer T\sphinxhyphen{}Shirt

\begin{sphinxVerbatim}[commandchars=\\\{\}]
\PYG{l+m+mi}{001} \PYG{p}{(}\PYG{n}{Rot}\PYG{p}{)}
\PYG{l+m+mi}{002} \PYG{p}{(}\PYG{n}{Gelb}\PYG{p}{)}
\PYG{l+m+mi}{003} \PYG{p}{(}\PYG{n}{Blau}\PYG{p}{)}
\PYG{l+m+mi}{004} \PYG{p}{(}\PYG{n}{Grün}\PYG{p}{)}
\end{sphinxVerbatim}

Varianten für Artikel Frauen T\sphinxhyphen{}Shirt

\begin{sphinxVerbatim}[commandchars=\\\{\}]
\PYG{l+m+mi}{001} \PYG{p}{(}\PYG{n}{Rot}\PYG{p}{)}
\PYG{l+m+mi}{002} \PYG{p}{(}\PYG{n}{Gelb}\PYG{p}{)}
\PYG{l+m+mi}{003} \PYG{p}{(}\PYG{n}{Blau}\PYG{p}{)}
\PYG{l+m+mi}{004} \PYG{p}{(}\PYG{n}{Grün}\PYG{p}{)}
\end{sphinxVerbatim}

\begin{sphinxadmonition}{attention}{Achtung:}
Die Hauptartikelnummer sollte auf jeden Fall eindeutig sein!
\end{sphinxadmonition}


\subparagraph{Größen}
\label{\detokenize{modules/general/skuscheme:groszen}}
Für Größen können die selben Schemas wie für Varianten verwendet werden.


\subparagraph{Mehrdimensionale Varianten}
\label{\detokenize{modules/general/skuscheme:mehrdimensionale-varianten}}
Oxid eShop bietet die Möglichkeit um mehrdimensionale Varianten abzubilden,
also Hauptartikel die in mehrfach verzweigte Subartikel haben.
Ein Beispiel für solche Varianten sind:
\begin{itemize}
\item {} 
T\sphinxhyphen{}Shirt, Farbe Rot, Größe M

\item {} 
Poloshirt, Farbe Blau, Größe XL

\item {} 
Tasse, Farbe Gold, Größe 150ml

\end{itemize}

Damit der Connector automatisch den zugrunde liegenden Shop Artikel ermitteln kann,
ist es notwendig dass Ihre Oxid Artikelnummern einem Schema folgen.
Dieses Schema sorgt dafür dass eine Subartikelnummer immer aus einer
Kombination von Haupt\sphinxhyphen{}, Varianten\sphinxhyphen{} und Größenartikelnummer besteht.


\subparagraph{Beispiele}
\label{\detokenize{modules/general/skuscheme:beispiele}}
Wie das Schema aussieht können Sie in den Einstellungen des Connectors selbst festlegen.

\begin{sphinxVerbatim}[commandchars=\\\{\}]
\PYG{p}{\PYGZob{}}\PYG{n}{PRODUCT\PYGZus{}SKU}\PYG{p}{\PYGZcb{}}\PYG{o}{\PYGZhy{}}\PYG{p}{\PYGZob{}}\PYG{n}{VARIANT\PYGZus{}SKU}\PYG{p}{\PYGZcb{}}\PYG{o}{\PYGZhy{}}\PYG{p}{\PYGZob{}}\PYG{n}{SIZE\PYGZus{}SKU}\PYG{p}{\PYGZcb{}}
\end{sphinxVerbatim}

Damit bei dieser Einstellung der richtige Artikel in den Warenkorb übertragen wird muss beispielsweise folgendes gelten:
\begin{itemize}
\item {} 
Der HBE Hauptartikel hat die Artikelnummer 000001

\item {} 
Die HBE Variante hat die Artikelnummer 001

\item {} 
Die HBE Größe hat die Artikelnummer 123

\item {} 
Die Mehrdimensionale Oxid Variante hat die Artikelnummer 000001\sphinxhyphen{}001\sphinxhyphen{}123

\end{itemize}

Sie können dieses Schema wie bereits erwähnt beliebig ändern, so könnten Sie auf Werte verzichten \sphinxhyphen{}
diese mit einem festen Prefix versehen oder weitere Einstellungen vornehmen.

So könnten Sie zum Beispiel mit dem Subnummern Artikelschema arbeiten wenn Sie folgende Einstellung setzen:

\begin{sphinxVerbatim}[commandchars=\\\{\}]
\PYG{p}{\PYGZob{}}\PYG{n}{VARIANT\PYGZus{}SKU}\PYG{p}{\PYGZcb{}}\PYG{o}{\PYGZhy{}}\PYG{p}{\PYGZob{}}\PYG{n}{SIZE\PYGZus{}SKU}\PYG{p}{\PYGZcb{}}
\end{sphinxVerbatim}

Damit bei dieser Einstellung der richtige Artikel in den Warenkorb übertragen wird muss beispielsweise folgendes gelten:
\begin{itemize}
\item {} 
Die HBE Variante hat die Artikelnummer 000001\sphinxhyphen{}001

\item {} 
Die HBE Größe hat die Artikelnummer 123

\item {} 
Die Mehrdimensionale Oxid Variante hat die Artikelnummer 000001\sphinxhyphen{}001\sphinxhyphen{}123

\end{itemize}


\subsubsection{Oxid eShop}
\label{\detokenize{modules/oxid/index:oxid-eshop}}\label{\detokenize{modules/oxid/index::doc}}

\paragraph{Installation}
\label{\detokenize{modules/oxid/install:installation}}\label{\detokenize{modules/oxid/install::doc}}
\begin{sphinxadmonition}{hint}{Hinweis:}
Das Shirtnetwork Modul ist nur mit Oxid eShop ab Version 6 kompatibel, frühere Versionen werden nicht unterstützt
\end{sphinxadmonition}

Die Installation des Moduls erfolgt via composer, für weitergehende Informationen zu Modulen im Oxid eShop rufen Sie bitte
das \sphinxhref{https://docs.oxid-esales.com/de/}{Handbuch des Shop Systems} auf.

Das Modulpaket wird über unser hausinternes Composer Repository verteilt, passen Sie die Datei composer.json im Shop Stammverzeichnis
wie folgt an:

\begin{sphinxVerbatim}[commandchars=\\\{\}]
\PYG{p}{\PYGZob{}}
  \PYG{n+nt}{\PYGZdq{}repositories\PYGZdq{}}\PYG{p}{:} \PYG{p}{[}\PYG{p}{\PYGZob{}}
    \PYG{n+nt}{\PYGZdq{}type\PYGZdq{}}\PYG{p}{:} \PYG{l+s+s2}{\PYGZdq{}composer\PYGZdq{}}\PYG{p}{,}
    \PYG{n+nt}{\PYGZdq{}url\PYGZdq{}}\PYG{p}{:} \PYG{l+s+s2}{\PYGZdq{}https://packages.aggrosoft.de\PYGZdq{}}
  \PYG{p}{\PYGZcb{}}\PYG{p}{]}
\PYG{p}{\PYGZcb{}}
\end{sphinxVerbatim}

Anschließend kann das Modul per composer installiert werden

\begin{sphinxVerbatim}[commandchars=\\\{\}]
composer require shirtnetwork/designer\PYGZhy{}oxid
\end{sphinxVerbatim}

Nach der Installation kann das Modul im Oxid eShop Backend aktiviert werden.


\paragraph{Konfiguration}
\label{\detokenize{modules/oxid/settings:konfiguration}}\label{\detokenize{modules/oxid/settings::doc}}
Das Modul kommt mit verschiedenen Einstellungen die vor der Verwendung konfiguriert werden müssen.


\subparagraph{Zugangsdaten}
\label{\detokenize{modules/oxid/settings:zugangsdaten}}
Geben Sie hier Ihren Shirtnetwork Benutzernamen und das Passwort ein, nachdem Sie die Einstellungen speichern wird
automatisch die zugehörige Benutzer ID ermittelt.


\subparagraph{Backend}
\label{\detokenize{modules/oxid/settings:backend}}
Geben Sie hier die URLs zu Ihrem {\hyperref[\detokenize{tools/backend::doc}]{\sphinxcrossref{\DUrole{doc}{Backend Server}}}} ein. \sphinxguilabel{https://backend.mein\sphinxhyphen{}shop.de/files}
für \sphinxcode{\sphinxupquote{Upload Server URL}} und \sphinxguilabel{https://backend.mein\sphinxhyphen{}shop.de} für \sphinxcode{\sphinxupquote{Config Server URL}}.

\sphinxcode{\sphinxupquote{EPS Tool Projekt}} enthält den Namen des {\hyperref[\detokenize{tools/epstool::doc}]{\sphinxcrossref{\DUrole{doc}{EPS Tool}}}} Projekts den Sie festgelegt haben.

\begin{sphinxadmonition}{attention}{Achtung:}
Achten Sie auf die genaue Schreibweise des Projektnamens, inklusive aller Leerzeichen und der Groß\sphinxhyphen{} und Kleinschreibung
\end{sphinxadmonition}


\subparagraph{Einstellungen}
\label{\detokenize{modules/oxid/settings:einstellungen}}

\subparagraph{Start Artikelnummer}
\label{\detokenize{modules/oxid/settings:start-artikelnummer}}
Gültige Artikelnummer die geladen werden soll wenn der Designer ohne eine Artikelnummer aufgerufen wird. Achten Sie unbedingt
darauf dass dieser Artikel aktiv ist und dem Shop zugewiesen wurde.


\subparagraph{Artikelnummern Schema}
\label{\detokenize{modules/oxid/settings:artikelnummern-schema}}
Ihr aktuelles Artikelnummern Schema, weitere Informationen hierzu entnehmen Sie bitte dem zugehörigen Handbucheintrag.


\subparagraph{Designer Version}
\label{\detokenize{modules/oxid/settings:designer-version}}
Version des Designers die geladen werden soll, das Feld schlägt automatisch die neuesten Versionen vor die zur Verfügung
stehen und zeigt diese an. Wählen Sie bei einer Neuinstallation die höchste Versionsnummer.


\subparagraph{Druckart Auswahlmodus}
\label{\detokenize{modules/oxid/settings:druckart-auswahlmodus}}
Legt fest wie Druckarten vom Benutzer gewählt werden können, für jedes einzelne Objekt oder nur pro Ansicht oder Produkt.
Bitte beachten Sie dass im \sphinxcode{\sphinxupquote{Nach Objekt}} Modus das Objekt (Text, Motiv) angewählt werden muss um die Druckart zu wechseln,
dies führt häufig zu Missverständnissen.


\subparagraph{Preise nicht überschreiben}
\label{\detokenize{modules/oxid/settings:preise-nicht-uberschreiben}}
Wenn diese Option aktiv ist werden immer die Preise aus dem Shopsystem verwendet.
Für die korrekte Preisanzeige im Designer ist dann das Shopsystem verantwortlich.


\subparagraph{Debug Modus}
\label{\detokenize{modules/oxid/settings:debug-modus}}
Aktiviert Entwickler Optionen, dadurch ist es möglich die \sphinxhref{https://github.com/vuejs/vue-devtools}{VueJS Entwickler Tools} zu verwenden.

\begin{sphinxadmonition}{attention}{Achtung:}
Verwenden Sie den Debug Modus niemals im Live Betrieb!
\end{sphinxadmonition}


\subparagraph{Produkte / Motive pro Seite}
\label{\detokenize{modules/oxid/settings:produkte-motive-pro-seite}}
Gibt an wie viele Produkte bzw. Motive pro Seite in den Listen im Designer dargestellt werden.


\subparagraph{Synchronisierung}
\label{\detokenize{modules/oxid/settings:synchronisierung}}
Mit diesen Einstellungen legen Sie fest wie das Oxid Feld \sphinxcode{\sphinxupquote{Name der Auswahl}} beim synchronisieren befüllt wird.
Mehr Informationen zu diesem Feld entnehmen Sie bitte dem \sphinxhref{https://docs.oxid-esales.com/eshop/de/6.2/einrichtung/artikel/registerkarte-varianten.html}{Oxid eShop Handbucheintrag}


\subparagraph{Daten Mapping}
\label{\detokenize{modules/oxid/settings:daten-mapping}}
Geben Sie hier Daten im Format \sphinxcode{\sphinxupquote{oxidfeld =\textgreater{} variable}} ein, im Designer Template können Sie auf diese Daten via \sphinxcode{\sphinxupquote{\$shop("variable")}} zugreifen.
So können Sie beliebige Oxid Datenbankfelder des Artikels im Designer verwenden z.B. \sphinxcode{\sphinxupquote{oxean =\textgreater{} ean}} macht die EAN im Designer Template
als \sphinxcode{\sphinxupquote{\$shop("ean")}} verfügbar.


\subparagraph{SEO}
\label{\detokenize{modules/oxid/settings:seo}}
Legen Sie hier die gewünschten Meta Tags für den Designer fest. Die Meta Angaben für Logo Kategorien geben Sie im Format \sphinxcode{\sphinxupquote{kategorieid =\textgreater{} Wert}} an.
\sphinxSetupCaptionForVerbatim{Beispiel für Logo Kategorie Meta}
\def\sphinxLiteralBlockLabel{\label{\detokenize{modules/oxid/settings:id1}}}
\begin{sphinxVerbatim}[commandchars=\\\{\}]
\PYG{x}{1001 =\PYGZgt{} Meine SEO description}
\PYG{x}{1002 =\PYGZgt{} Eine andere Beschreibung}
\PYG{x}{1004 =\PYGZgt{} Noch mehr}
\end{sphinxVerbatim}

Die ID’s der einzelnen Kategorien lesen Sie am einfachsten aus der URL der Logoliste ab, diese finden Sie unter \sphinxurl{https://www.ihr-shop.de/index.php?cl=designer\_logos}


\subparagraph{Laufzettel}
\label{\detokenize{modules/oxid/settings:laufzettel}}
Hier finden Sie verschiedene Einstellungen mit denen Sie die Ausgabe des Laufzettels zu den Bestellungen anpassen können.


\paragraph{Verwendung}
\label{\detokenize{modules/oxid/usage:verwendung}}\label{\detokenize{modules/oxid/usage::doc}}

\subparagraph{Artikel synchronisieren}
\label{\detokenize{modules/oxid/usage:artikel-synchronisieren}}
Damit Artikel in den Warenkorb gelegt werden können müssen diese mit einer {\hyperref[\detokenize{modules/general/skuscheme::doc}]{\sphinxcrossref{\DUrole{doc}{passenden Artikelnummer}}}} in Ihrem Shop existieren.
Um doppelte Datenpflege zu vermeiden ist es möglich Artikel vom HBE in Ihren Shop zu übernehmen.
Bei diesem Vorgang wird das aktuelle {\hyperref[\detokenize{modules/general/skuscheme::doc}]{\sphinxcrossref{\DUrole{doc}{Artikelnummern Schema}}}} und die {\hyperref[\detokenize{modules/oxid/settings::doc}]{\sphinxcrossref{\DUrole{doc}{Einstellungen zur Synchronisierung}}}} beachtet.

Sie finden die Synchronisierung im Admin Bereich unter \sphinxcode{\sphinxupquote{Shirtnetwork =\textgreater{} Synchronisation}}. Wählen Sie den gewünschten
Artikel aus und drücken Sie \sphinxguilabel{Artikel synchronisieren} um den Artikel anzulegen.

\begin{sphinxadmonition}{attention}{Achtung:}
Sollte der Artikel bereits vorher synchronisiert worden sein werden eventuelle Änderungen überschrieben.
\end{sphinxadmonition}


\subparagraph{Verlinkungen}
\label{\detokenize{modules/oxid/usage:verlinkungen}}
Das Modul stellt aktuell 2 Oberflächen bereit die im Frontend verlinkt werden können. Dies kann über einen externen Link
einer Kategorie erfolgen, oder über einen Link innerhalb des Templates oder eines CMS Inhaltes.


\subparagraph{Designer}
\label{\detokenize{modules/oxid/usage:designer}}
Zeigt den eigentlichen Shirtnetwork Designer an. Der Link lautet wie folgt:

\begin{sphinxVerbatim}[commandchars=\\\{\}]
https://www.mein\PYGZhy{}shop.de/index.php?cl=designer
\end{sphinxVerbatim}

Folgende zusätzliche Parameter können wahlweise übergeben werden um beliebige Daten vorzugeben:


\begin{savenotes}\sphinxattablestart
\centering
\begin{tabulary}{\linewidth}[t]{|T|T|}
\hline
\sphinxstyletheadfamily 
Parameter
&\sphinxstyletheadfamily 
Funktion
\\
\hline
\sphinxcode{\sphinxupquote{artnr}}
&
Artikelnummer des Hauptartikels
\\
\hline
\sphinxcode{\sphinxupquote{vartnr}}
&
Artikelnummer der Variante
\\
\hline
\sphinxcode{\sphinxupquote{sartnr}}
&
Artikelnummer der Größe
\\
\hline
\sphinxcode{\sphinxupquote{config}}
&
ID einer Konfiguration die auf dem Backend Server liegt (nicht kompatibel mit anderen Parametern)
\\
\hline
\sphinxcode{\sphinxupquote{text}}
&
Start Text
\\
\hline
\sphinxcode{\sphinxupquote{logo}}
&
ID eines Logos
\\
\hline
\sphinxcode{\sphinxupquote{overrides}}
&
Alternativer Template Override
\\
\hline
\sphinxcode{\sphinxupquote{custom}}
&
Alternatives zusätzliches Javascript
\\
\hline
\sphinxcode{\sphinxupquote{keep}}
&
Wenn Wert = 1 dann werden bereits gestaltete Inhalte des Kunden nicht gelöscht
\\
\hline
\sphinxcode{\sphinxupquote{plain}}
&
Wenn Wert = 1 aktiviert den Popup Modus (nur der Designer wird gezeigt, ohne Shop layout)
\\
\hline
\end{tabulary}
\par
\sphinxattableend\end{savenotes}


\subparagraph{Motive}
\label{\detokenize{modules/oxid/usage:motive}}
Zeigt die vorhandenen Motive und Motivkategorien an. Der Link lautet wie folgt:

\begin{sphinxVerbatim}[commandchars=\\\{\}]
https://www.mein\PYGZhy{}shop.de/index.php?cl=designer\PYGZus{}logos
\end{sphinxVerbatim}

Der Parameter \sphinxcode{\sphinxupquote{lcid}} kann gesetzt werden um eine beliebige Logo Kategorie zu laden.


\paragraph{Bestellungen bearbeiten}
\label{\detokenize{modules/oxid/orders:bestellungen-bearbeiten}}\label{\detokenize{modules/oxid/orders::doc}}
Das Modul erweitert den Reiter \sphinxcode{\sphinxupquote{Artikel}} der Bestellübersicht. Sie finden den Punkt im Admin Bereich unter
\sphinxcode{\sphinxupquote{Bestellungen verwalten =\textgreater{} Bestellungen}}.

\begin{figure}[htbp]
\centering
\capstart

\noindent\sphinxincludegraphics{{orders}.png}
\caption{Artikel Details einer Bestellung}\label{\detokenize{modules/oxid/orders:id1}}\end{figure}

Es werden Ihnen alle Details der vom Kunden gestalteten Artikel angezeigt. Mit dem Button \sphinxguilabel{Laufzettel drucken}
erzeugen Sie eine Auftragsübersicht zur Produktion.


\subparagraph{Druckdateien generieren}
\label{\detokenize{modules/oxid/orders:druckdateien-generieren}}
Starten Sie das {\hyperref[\detokenize{tools/epstool::doc}]{\sphinxcrossref{\DUrole{doc}{EPS Tool}}}} bevor Sie den die Artikelübersicht im Browser aufrufen. Prüfen Sie nun
ob der Status \sphinxcode{\sphinxupquote{EPS Tool Verbindung}} auf \sphinxcode{\sphinxupquote{Verbunden}} steht. Sollte die Verbindung nicht aufgebaut sein drücken Sie
den Button \sphinxguilabel{Verbindung prüfen} und bestätigen Sie eventuelle Sicherheitsmeldungen mit einer Ausnahme.
Rufen Sie anschließend die Artikelübersicht erneut auf.

\begin{figure}[htbp]
\centering
\capstart

\noindent\sphinxincludegraphics{{epstool-oxid-status}.png}
\caption{Statusmeldung EPS Tool}\label{\detokenize{modules/oxid/orders:id2}}\end{figure}

\sphinxguilabel{Druckdateien generieren} übergibt die geforderte Bestellung automatisch an das EPS Tool und öffnet die Datei.
Prüfen Sie den Status im EPS Tool, die Erstellung kann einige Sekunden dauern.


\section{Bestellungen verarbeiten}
\label{\detokenize{intro/orders:bestellungen-verarbeiten}}\label{\detokenize{intro/orders::doc}}
Wenn eine Bestellung in Ihrem System eingeht können Sie die Personalisierungs Details einsehen und mit Hilfe des sog.
EPS\sphinxhyphen{}Tool druckfertige Daten generieren.

Installieren Sie bitte das {\hyperref[\detokenize{tools/epstool::doc}]{\sphinxcrossref{\DUrole{doc}{EPS\sphinxhyphen{}Tool entsprechend des Handbucheintrages}}}}, wie Sie die Bestelldaten aus dem Shop erstellen
wird näher im Handbucheintrag zum jeweiligen Connector erklärt.


\chapter{Tools}
\label{\detokenize{tools/index:tools}}\label{\detokenize{tools/index::doc}}

\section{Backend Server}
\label{\detokenize{tools/backend:backend-server}}\label{\detokenize{tools/backend::doc}}
Der Backend Server speichert die Kundenuploads und Konfigurationen, außerdem stellt er die Schriften für den Designer
bereit. Detailinformationen zur Software entnehmen Sie bitte unserem zugehörigen \sphinxhref{https://github.com/aggrosoft/designer-backend-skeleton}{Github Repository}.


\subsection{Installation}
\label{\detokenize{tools/backend:installation}}
Für die Installation des Backend Servers ist \sphinxhref{https://git-scm.com/}{git} und \sphinxhref{https://getcomposer.org/doc/00-intro.md}{Composer} notwendig,
installieren zunächst \sphinxhref{https://git-scm.com/}{git} und \sphinxhref{https://getcomposer.org/doc/00-intro.md}{Composer} nach Anleitung auf der Website.

Bauen Sie eine SSH Verbindung zu Ihrem Server auf und führen Sie folgende Befehle nacheinander aus:

\begin{sphinxVerbatim}[commandchars=\\\{\}]
git clone https://github.com/aggrosoft/designer\PYGZhy{}backend\PYGZhy{}skeleton.git
\PYG{n+nb}{cd} designer\PYGZhy{}backend\PYGZhy{}skeleton
composer install
cp cfg.inc.sample.php cfg.inc.php
\end{sphinxVerbatim}

Passen Sie die Datei \sphinxcode{\sphinxupquote{cfg.inc.php}} nach Ihren Wünschen an, wir raten dazu mindestens die Einstellung \sphinxcode{\sphinxupquote{auth =\textgreater{} users}}
zu setzen. Hierbei handelt es sich um Benutzernamen und Kennwort um die Einstellungen des Servers zu verändern.

Erstellen Sie nun eine Subdomain die auf den Installationsordner des Backend Servers zeigt e.g. \sphinxcode{\sphinxupquote{https://backend.mein\sphinxhyphen{}shop.de}}

Wenn die Subdomain fertig eingerichtet ist können Sie unter \sphinxcode{\sphinxupquote{https://backend.mein\sphinxhyphen{}shop.de/settings/fonts}} die Schriftarten
konfigurieren. Anschließend ist die Software einsatzbereit.


\subsection{Update}
\label{\detokenize{tools/backend:update}}
Aktualisierungen werden ebenfalls per composer vorgenommen, führen Sie hierfür einfach folgenden Befehl im Verzeichnis aus:

\begin{sphinxVerbatim}[commandchars=\\\{\}]
composer update
\end{sphinxVerbatim}


\section{EPS Tool}
\label{\detokenize{tools/epstool:eps-tool}}\label{\detokenize{tools/epstool::doc}}
Das EPS Tool erzeugt aus den Konfigurationen Ihrer Kunden automatisch Druckdaten.
Das Tool kommuniziert hierfür mit dem {\hyperref[\detokenize{tools/backend::doc}]{\sphinxcrossref{\DUrole{doc}{Backend Server}}}}, dieser sollte also vorher installiert worden sein.


\subsection{Installation}
\label{\detokenize{tools/epstool:installation}}

\subsubsection{Java installieren}
\label{\detokenize{tools/epstool:java-installieren}}
Das EPS Tool basiert auf Java, um es zu verwenden muss ein sog. Java Runtime Environment (JRE) auf Ihrem
Computer installiert sein. Wir empfehlen den Einsatz des kostenfreien Open JDK von Oracle.

Adopt Open JDK stellt fertige Installationpakete für Open JDK bereit, dies vereinfacht die Installation.

Laden Sie das \sphinxhref{https://adoptopenjdk.net}{Open JDK Installationspaket} für ihr Betriebssystem herunter und führen Sie es aus.
Die Vorauswahl auf der Website ist bereits optimal eingestellt. Folgen Sie den Schritten im Installer, weitere Hilfe finden Sie auf der Website von Adopt Open JDK.


\subsubsection{Paket installieren}
\label{\detokenize{tools/epstool:paket-installieren}}
Laden Sie zunächst das Installationspaket aus unserem \sphinxhref{https://www.shirtnetwork.de/downloads}{Download Bereich} herunter.
Das Paket kommt als .zip Archiv und muss entpackt werden, im Normalfall bietet ihr Betriebssystem hierfür eine Möglichkeit.
Entpacken Sie das Archiv in ein Verzeichnis Ihrer Wahl z.B. \sphinxguilabel{C:\textbackslash{}epstool}. \sphinxstylestrong{Der Verzeichnisname darf keine runden oder eckigen Klammern enthalten,
er darf auch keine Umlaute enthalten.}


\subsubsection{Perl installieren}
\label{\detokenize{tools/epstool:perl-installieren}}
Unter Windows ist es zusätzlich notwendig Perl zu installieren, laden Sie das \sphinxhref{http://strawberryperl.com/}{Strawberry Perl Installationspaket} herunter
und installieren Sie es unter \sphinxguilabel{C:\textbackslash{}strawberry} (Standardeinstellung)


\subsubsection{Software starten}
\label{\detokenize{tools/epstool:software-starten}}
Starten Sie das EPS Tool per Doppelklick auf die Datei \sphinxguilabel{PdfLogoCreator.jar} im Installationsordner des EPS Tools.

Sollte das Tool nicht starten führen Sie bitte den folgenden Befehl auf der Kommandozeile Ihres Betriebssystems aus

\sphinxguilabel{java \sphinxhyphen{}jar C:\textbackslash{}epstool\textbackslash{}PDFLogoCreator.jar}

Passen Sie den Pfad an ihren gewählten Installationsordner an, achten Sie auf eventuelle Fehlermeldung die ausgegeben
werden. Sollten Sie die Fehlermeldung nicht selbst beheben können senden Sie bitte die vollständige Ausgabe an unseren Support.


\subsection{Konfiguration}
\label{\detokenize{tools/epstool:konfiguration}}
Nach dem ersten Start muss ein Projekt konfiguriert werden, der Bildschirm hierfür wird automatisch geöffnet.

\begin{figure}[htbp]
\centering
\capstart

\noindent\sphinxincludegraphics{{epstool-project}.png}
\caption{Projektbildschirm EPS Tool}\label{\detokenize{tools/epstool:id1}}\end{figure}

Füllen Sie unter \sphinxcode{\sphinxupquote{or create new project}} die Felder aus.
\begin{itemize}
\item {} 
\sphinxstylestrong{Project:} Name des Projekts, frei definierbar. Halten Sie den Namen am besten kurz und prägnant.

\item {} 
\sphinxstylestrong{Shop\sphinxhyphen{}URL:} URL zum {\hyperref[\detokenize{tools/backend::doc}]{\sphinxcrossref{\DUrole{doc}{Backend Server}}}} e.g. \sphinxguilabel{https://backend.mein\sphinxhyphen{}shop.de}

\item {} 
\sphinxstylestrong{HBE\sphinxhyphen{}Login:} Shirtnetwork Benutzername bzw. E\sphinxhyphen{}Mail Adresse

\item {} 
\sphinxstylestrong{HBE\sphinxhyphen{}Password:} Shirtnetwork Passwort

\item {} 
\sphinxstylestrong{Shop System:} Wählen Sie in diesem Fall bitte unbedingt \sphinxcode{\sphinxupquote{Config\sphinxhyphen{}Server}}

\end{itemize}

Drücken Sie auf OK um das neue Projekt zu speichern.


\subsection{Verwendung}
\label{\detokenize{tools/epstool:verwendung}}
Das EPS Tool erlaubt es Ihnen Bestellungen manuell zu filtern und zu generieren, zusätzlich lässt sich die Generierung
eines bestimmten Auftrags auch direkt aus der Shop bzw. Warexo Backend starten.

\begin{figure}[htbp]
\centering
\capstart

\noindent\sphinxincludegraphics{{epstool-list}.png}
\caption{EPS Tool Hauptbildschirm}\label{\detokenize{tools/epstool:id2}}\end{figure}

Geben Sie unter \sphinxguilabel{from} und \sphinxguilabel{until} einen Datumsbereich ein oder tragen Sie unter \sphinxguilabel{Orders} eine
oder mehrere Bestellnummern (Komma getrennt) ein und klicken Sie auf \sphinxguilabel{Load Orders} um eine Liste mit Bestellungen
zu erstellen.

Wählen Sie per Rechtsklick einen Eintrag aus, der Befehl \sphinxguilabel{Generate EPS/AI/PDF} erzeugt die Druckdaten. Nachdem
der Vorgang abgeschlossen ist können Sie per Rechtsklick \sphinxguilabel{Open …} die Druckdatei im gewünschten Format öffnen.


\chapter{Preismodell}
\label{\detokenize{pricing/index:preismodell}}\label{\detokenize{pricing/index::doc}}

\section{Gebühren}
\label{\detokenize{pricing/fees:gebuhren}}\label{\detokenize{pricing/fees::doc}}
Für die Nutzung von Shirtnetwork fallen Gebühren an, Sie zahlen jedoch nur für Artikel die Sie wirklich verkauft haben.
Die Gebühren werden auf zwei Arten verrechnet, entweder per Lastschrift Verfahren für Kunden mit einem Gebührenvolumen von mindestens 300\texteuro{} monatlich, oder per Prepaid (Coins) System.
Für das Lastschrift Verfahren müssen Sie sich bei Interesse seperat anmelden, kontaktieren Sie uns hierfür einfach direkt.

\sphinxstyleemphasis{Da die Gebühren auf den Brutto Preis anfallen sind alle Preisangaben inklusive Mehrwertsteuer}


\subsection{Berechnung}
\label{\detokenize{pricing/fees:berechnung}}
Für Artikel mit einem brutto Verkaufswert über 10\texteuro{} zahlen Sie pauschal 1\texteuro{} pro verkauftem Artikel
Für Artikel mit einem brutto Verkaufswert unter 10\texteuro{} zahlen Sie 10\% des Verkaufswertes


\subsubsection{Beispiel}
\label{\detokenize{pricing/fees:beispiel}}
Sie verkaufen über Ihren Online Shop 3 Artikel mit Konfiguration, 1 Artikel für 10,90\texteuro{} und 2 Artikel für 5,20\texteuro{} \sphinxhyphen{} es fallen nun folgende Gebühren an:
\begin{itemize}
\item {} 
1 x 1,00\texteuro{} für den Artikel mit VK 10,90\texteuro{}

\item {} 
2 x 0,52\texteuro{} für die Artikel mit VK 5,20\texteuro{}

\item {} 
= 2,04\texteuro{} Gebühren

\end{itemize}

Für Artikel die nicht konfiguriert wurden zahlen Sie keine Gebühren!


\subsection{Preisstaffeln}
\label{\detokenize{pricing/fees:preisstaffeln}}
Bei entsprechend hohem Verkaufvolumen erhalten Sie einen Rabatt auf Ihre Gebühren.


\subsubsection{Prepaid}
\label{\detokenize{pricing/fees:prepaid}}

\begin{savenotes}\sphinxattablestart
\centering
\begin{tabulary}{\linewidth}[t]{|T|T|T|}
\hline
\sphinxstyletheadfamily 
Coins
&\sphinxstyletheadfamily 
Kosten
&\sphinxstyletheadfamily 
Rabatt
\\
\hline
10
&
10,00\texteuro{}
&
n.V.
\\
\hline
20
&
20,00\texteuro{}
&
n.V.
\\
\hline
50
&
50,00\texteuro{}
&
n.V.
\\
\hline
100
&
100,00\texteuro{}
&
n.V.
\\
\hline
250
&
245,00\texteuro{}
&
5,00\texteuro{}
\\
\hline
500
&
490,00\texteuro{}
&
10,00\texteuro{}
\\
\hline
1000
&
975,00\texteuro{}
&
25,00\texteuro{}
\\
\hline
2500
&
2000,00\texteuro{}
&
500,00\texteuro{}
\\
\hline
5000
&
4000,00\texteuro{}
&
1000,00\texteuro{}
\\
\hline
\end{tabulary}
\par
\sphinxattableend\end{savenotes}


\subsubsection{Lastschrift}
\label{\detokenize{pricing/fees:lastschrift}}

\begin{savenotes}\sphinxattablestart
\centering
\begin{tabulary}{\linewidth}[t]{|T|T|}
\hline
\sphinxstyletheadfamily 
Verkaufte Artikel monatlich
&\sphinxstyletheadfamily 
Gebühr
\\
\hline
ab 2500
&
0,75\texteuro{} / 7,5\%
\\
\hline
ab 5000
&
0,50\texteuro{} / 5,0\%
\\
\hline
\end{tabulary}
\par
\sphinxattableend\end{savenotes}


\chapter{HBE \sphinxhyphen{} Händler Backend}
\label{\detokenize{hbe/index:hbe-handler-backend}}\label{\detokenize{hbe/index::doc}}
Da Shirtnetwork Artikel mehr Daten als ein klassischer Artikel benötigen wie z.B. verschiedene Ansichten, Druckarten etc.
ist es notwendig die Artikeldaten in einer separaten Software anzulegen. Die Daten können massenweise importiert
oder manuell eingegeben werden, im Anschluss können die notwendigen Daten in Ihr Shopsystem übertragen werden.

Die Verwaltungssoftware HBE finden Sie online, sie lässt sich in den meisten modernen Browsern bedienen und ist Endgeräte unabhängig.
Sie finden das HBE unter \sphinxurl{https://hbe.shirtnetwork.de}


\section{Login}
\label{\detokenize{hbe/login:login}}\label{\detokenize{hbe/login::doc}}
Sie finden die Software immer aktuell unter \sphinxurl{https://hbe.shirtnetwork.de}

Der erste Schritt zur Verwaltung Ihrer Produkte ist der Login Bereich. Nach dem Start der Software sehen Sie den Login
Bildschirm vor sich, hier geben Sie bitte Ihre Shirtnetwork Zugangsdaten ein.

Wenn Sie sich nicht einloggen können fordern Sie über unsere Homepage ein neues Kennwort an oder
wenden Sie sich an unseren Support.


\section{Übersicht}
\label{\detokenize{hbe/intro:ubersicht}}\label{\detokenize{hbe/intro::doc}}
Das HBE ist in verschiedene Bereiche aufgegliedert:

\begin{figure}[htbp]
\centering
\capstart

\noindent\sphinxincludegraphics{{hbe-screen-overview}.png}
\caption{HBE Übersicht}\label{\detokenize{hbe/intro:id1}}\end{figure}

Auf der linken Seite befindet sich die Menüleiste, hier finden Sie Links zu den einzelnen Funktionen.
Je nach gewählter Funktion sehen Sie rechts die zugehörige Liste bzw. Formular


\subsection{Listen}
\label{\detokenize{hbe/intro:listen}}
Die Listen innerhalb des HBE bieten unterschiedliche Funktionen:

\begin{figure}[htbp]
\centering
\capstart

\noindent\sphinxincludegraphics{{hbe-lists}.png}
\caption{HBE Listen}\label{\detokenize{hbe/intro:id2}}\end{figure}

\begin{sphinxShadowBox}
\sphinxstylesidebartitle{Neuer Datensatz}

\begin{center}
\noindent\sphinxincludegraphics{{hbe-lists-add}.png}
\end{center}
\end{sphinxShadowBox}

Sie sehen den Titel der Liste, eventuell eine Kategorieauswahl und eine Sucheingabe.
Das große rote Plus Symbol \sphinxstyleemphasis{} rechts oben öffnet das Formular für einen neuen Datensatz.

Ein klick auf einen Spaltentitel sortiert die Liste nach dieser Spalte, erneutes drücken
dreht die Sortierung um.

Die erste Spalte erlaubt es mehrere Datensätze anzuwählen, dadurch erscheint im Fuß der Tabelle die Massenbearbeitung.
Hiermit lassen sich die Einträge löschen oder als CSV Datei exportieren.

Die letzte Spalte der Liste enthält die Funktionen für den jeweiligen Datensatz, wie z.B. bearbeiten, löschen,
duplizieren oder konfigurieren. Fahren Sie mit der Maus über das Symbol um zu sehen welche Funktion ausgeführt wird.
Wenn Sie versuchen einen Datensatz zu löschen oder zu kopieren wird vorher sicherheitshalber eine Abfrage angezeigt.

Im Fuß der Tabelle können Sie wählen wie viele Datensätze angezeigt werden und zwischen den Seiten blättern.


\subsection{Formulare}
\label{\detokenize{hbe/intro:formulare}}
\begin{sphinxShadowBox}
\sphinxstylesidebartitle{Formular Reiter}

\begin{center}
\noindent\sphinxincludegraphics{{hbe-form-tabs}.png}
\end{center}
\end{sphinxShadowBox}

Wenn Sie einen Datensatz erzeugen oder bearbeiten wird ein Eingabeformular in einem Fenster angezeigt. Die einzelnen
Felder werden im weiteren Handbuch erklärt. Formulare können in mehrere Reiter eingeteilt sein, diese werden am oberen
Rand des Formulars dargestellt. Klicken Sie auf einen Reiter um ihn zu aktivieren.
Am unteren Ende finden Sie die \sphinxguilabel{Speichern} sowie \sphinxguilabel{Abbruch} Buttons.


\section{Ersteinrichtung}
\label{\detokenize{hbe/setup:ersteinrichtung}}\label{\detokenize{hbe/setup::doc}}
Bevor Sie Artikel oder Motive anlegen sollten Sie zunächst einige grundlegende Daten anlegen. Benötigt werden:
\begin{itemize}
\item {} 
Druckarten

\item {} 
Steuersätze

\item {} 
Größen

\item {} 
Produktkategorien

\item {} 
Motivkategorien

\end{itemize}

Hierfür bietet das HBE Ihnen die Möglichkeit die benötigten Einträge über einen Ersteinrichtungs Assistenten zu erledigen.
Sie finden den Assistenten in der Menüleiste unter \sphinxstyleemphasis{} \sphinxcode{\sphinxupquote{Ersteinrichtung}}.

Alternativ können Sie die Datensätze auch direkt über die jeweiligen Menüpunkte erzeugen.


\section{Druckarten}
\label{\detokenize{hbe/printtypes:druckarten}}\label{\detokenize{hbe/printtypes::doc}}
Druckarten bestimmen die verschiedenen Arten der Personalisierung die Sie anbieten möchten
wie z.B. Digitaldruck, Stickerei, Gravur etc.


\subsection{Eingabefelder}
\label{\detokenize{hbe/printtypes:eingabefelder}}

\subsubsection{Aktiv}
\label{\detokenize{hbe/printtypes:aktiv}}
Nur aktive Druckarten stehen zur Auswahl


\subsubsection{Digitaldruckart}
\label{\detokenize{hbe/printtypes:digitaldruckart}}
Bei Digitaldruckarten können die Kunden aus beliebigen Farben wählen,
sie müssen in diesem Fall nicht jede Farbe einzeln definieren.


\subsubsection{Upload erlaubt}
\label{\detokenize{hbe/printtypes:upload-erlaubt}}
Wenn aktiviert kann der Kunde eigene Bilder hochladen und mit dieser Druckart
verwenden.


\subsubsection{Titel}
\label{\detokenize{hbe/printtypes:titel}}
Angezeigter Titel dieser Druckart


\subsubsection{Sortierung}
\label{\detokenize{hbe/printtypes:sortierung}}
Druckarten werden anhand dieses Feldes sortiert, von niedrig nach hoch.


\subsubsection{cm\(\sp{\text{2}}\) Preis}
\label{\detokenize{hbe/printtypes:cm2-preis}}
Wenn bei einem Produkt die cm\(\sp{\text{2}}\) Berechnung aktiviert ist wird dieser Wert
verwendet um den Aufpreis zu ermitteln. Die Kosten des Objekts sind dann
cm\(\sp{\text{2}}\) multipliziert mit diesem Wert.


\subsubsection{Aufschlag Upload}
\label{\detokenize{hbe/printtypes:aufschlag-upload}}
Zusatzkosten für jeden Upload der diese Druckart verwenden \sphinxhyphen{}
gilt für jeden Artikel wenn der Kunde mehr als einen Artikel bestellt


\subsubsection{Aufschlag Sonderfarbe}
\label{\detokenize{hbe/printtypes:aufschlag-sonderfarbe}}
Aktuell nicht verwendet


\subsubsection{Sonderaufschlag Upload}
\label{\detokenize{hbe/printtypes:sonderaufschlag-upload}}
Zusatzkosten für jeden Upload der diese Druckart verwenden \sphinxhyphen{}
gilt nur einmalig egal wie viele Artikel der Kunde bestellt (z.B. für Vorkosten Punching Stick)


\subsubsection{Erlaubte Schriftarten}
\label{\detokenize{hbe/printtypes:erlaubte-schriftarten}}
Wenn Sie keine erlaubten Schriftarten wählen sind alle Schriftarten für diese Druckart erlaubt.
Die Suchvorschläge entsprechen nicht unbedingt den verfügbaren Schriftarten.


\subsubsection{Beschreibung}
\label{\detokenize{hbe/printtypes:beschreibung}}
HTML Beschreibung der Druckart, kann im Designer Template verwendet werden.


\subsubsection{Farb Preisstaffel}
\label{\detokenize{hbe/printtypes:farb-preisstaffel}}
Aufpreis für diese Druckart anhand der Anzahl der verwendeten
Farben und der vom Kunden gewählten Menge. Hinzufügen über den
großen roten Plus Button, entfernen über das Mülltonnensymbol am
Ende der Zeile.


\subsubsection{cm\(\sp{\text{2}}\) Preisstaffel}
\label{\detokenize{hbe/printtypes:cm2-preisstaffel}}
Aufpreis für diese Druckart anhand der cm\(\sp{\text{2}}\) Größe des Objekts.
Hinzufügen über den großen roten Plus Button,
entfernen über das Mülltonnensymbol am
Ende der Zeile.


\subsection{Farben}
\label{\detokenize{hbe/printtypes:farben}}
\begin{sphinxadmonition}{attention}{Achtung:}
Eine Druckart muss mindestens eine Farbe besitzen um verwendet zu werden,
\sphinxstylestrong{dies gilt auch für Digitaldruckarten}.
\end{sphinxadmonition}

Jede Druckart erlaubt es dem Kunden die Objekte im Designer einzufärben (sofern das Objekt färbbar ist).
Hierzu muss jede wählbare Farbe angelegt werden.


\subsubsection{Eingabefelder}
\label{\detokenize{hbe/printtypes:id1}}

\paragraph{Aktiv}
\label{\detokenize{hbe/printtypes:id2}}
Nur aktive Farben werden angezeigt.


\paragraph{Titel}
\label{\detokenize{hbe/printtypes:id3}}
Angezeigter Titel der Farbe


\paragraph{Sortierung}
\label{\detokenize{hbe/printtypes:id4}}
Die Farben werden anhand dieser Sortierung sortiert, aufsteigend von niedrig nach hoch.


\paragraph{Farbwahl}
\label{\detokenize{hbe/printtypes:farbwahl}}
Wählen Sie im Farbwähler die Farbe in der die Objekte eingefärbt werden. Alternativ können Sie auch den gewünschten
Hex\sphinxhyphen{}Wert im Eingabefeld direkt eingeben.


\paragraph{Bild}
\label{\detokenize{hbe/printtypes:bild}}
Bestimmte Effektfarben wie z.B. Gold, Silber etc. sollen im Designer mit einem Farbbild statt mit einer Vollfarbe
angezeigt werden. Laden Sie hier das gewünschte Bild hoch.


\section{Steuersätze}
\label{\detokenize{hbe/taxes:steuersatze}}\label{\detokenize{hbe/taxes::doc}}
Da Produkte verschiedenen Steuersätzen unterliegen können bzw. sich der Hauptsteuersatz ändern kann ist es notwendig
mindestens einen Steuersatz anzulegen. Diese können Sie dann beliebig den Produkten zuweisen.


\subsection{Eingabefelder}
\label{\detokenize{hbe/taxes:eingabefelder}}

\subsubsection{Titel}
\label{\detokenize{hbe/taxes:titel}}
Titel des Steuersatzes, dient zur internen Kennzeichnung z.B. DE


\subsubsection{Wert}
\label{\detokenize{hbe/taxes:wert}}
Mehrwertsteuerwert in \% \sphinxhyphen{} für Deutschland aktuell 19


\section{Produkte}
\label{\detokenize{hbe/products/index:produkte}}\label{\detokenize{hbe/products/index::doc}}

\subsection{Produktkategorien}
\label{\detokenize{hbe/products/categories:produktkategorien}}\label{\detokenize{hbe/products/categories::doc}}
Um Produkte organisieren zu können müssen Sie zunächst einen Kategoriebaum anlegen. Dieser kann dem in Ihrem
Online Shop oder auch völlig anders aufgebaut sein.


\subsubsection{Eingabefelder}
\label{\detokenize{hbe/products/categories:eingabefelder}}

\paragraph{Aktiv}
\label{\detokenize{hbe/products/categories:aktiv}}
Nur aktive Kategorien werden angezeigt


\paragraph{Titel}
\label{\detokenize{hbe/products/categories:titel}}
Angezeigter Titel dieser Kategorie


\paragraph{Sortierung}
\label{\detokenize{hbe/products/categories:sortierung}}
Die Kategorien werden anhand dieses Wertes sortiert, aufsteigend von niedrig nach hoch.


\paragraph{Elternkategorie}
\label{\detokenize{hbe/products/categories:elternkategorie}}
Die Oberkategorie dieser Kategorie, wenn es sich um eine Hauptkategorie handelt wählen Sie bitte \sphinxcode{\sphinxupquote{\sphinxhyphen{}\sphinxhyphen{}\sphinxhyphen{}}}


\subsection{Größen}
\label{\detokenize{hbe/products/sizes:groszen}}\label{\detokenize{hbe/products/sizes::doc}}
Produkte können verschiedene Größen besitzen (z.B. für Textilien S,M,L). Damit Sie die Größen nicht für jedes
Produkt neu definieren müssen werden diese hier zentral verwaltet und später dem Produkt zugewiesen.

Eine Größenangabe ist nicht zwingend erforderlich, wenn Ihre Produkte nicht über verschiedene Größen verfügen,
können Sie diesen Schritt überspringen.

Es müssen hier alle möglichen Größen erstellt werden, unabhängig von einem Produkt. Wenn Sie also z.B.
T\sphinxhyphen{}Shirts und Tassen als Produkte haben, müssen hier sowohl die T\sphinxhyphen{}Shirt Größen als auch die Größen für die
Tassen erstellt werden.


\subsection{Produkte}
\label{\detokenize{hbe/products/products:produkte}}\label{\detokenize{hbe/products/products::doc}}
Produkte sind der zentrale Bestandteil des Designers und enthalten alle Informationen die zur Personalisierung notwendig
sind. Produkte bestehen aus Varianten, und Varianten wiederrum aus Ansichten.


\subsubsection{Eingabefelder}
\label{\detokenize{hbe/products/fields/index:eingabefelder}}\label{\detokenize{hbe/products/fields/index::doc}}

\paragraph{Stammdaten}
\label{\detokenize{hbe/products/fields/base:stammdaten}}\label{\detokenize{hbe/products/fields/base::doc}}

\subparagraph{Aktiv}
\label{\detokenize{hbe/products/fields/base:aktiv}}
Sie können den Artikel „An“ bzw. „Aus“ schalten. Ist er
nicht aktiv, so erscheint der Artikel auch nicht im Designer.


\subparagraph{Titel}
\label{\detokenize{hbe/products/fields/base:titel}}
Bezeichnung des Artikels.


\subparagraph{Kategorie}
\label{\detokenize{hbe/products/fields/base:kategorie}}
Gewünschte Kategorie in der dieser Artikel auftauchen soll.


\subparagraph{Artikelnummer}
\label{\detokenize{hbe/products/fields/base:artikelnummer}}
Die gewünschte Artikelnummer des Artikels.
Es ist dabei unbedingt darauf zu achten, dass die
Artikelnummer sowohl im HBE als auch in ihrem
Shop System identisch ist.


\subparagraph{Sortierung}
\label{\detokenize{hbe/products/fields/base:sortierung}}
gibt die Reihenfolge der Artikel innerhalb einer
Kategorie an.


\subparagraph{Bild}
\label{\detokenize{hbe/products/fields/base:bild}}
Das Produktbild des Artikels. i.d.R. ist das ein normales
Produktfoto, wie Sie es in ihrem Shop ebenfalls benötigen.
Dieses Bild wird in der Auswahlliste der Produkte im
Designer verwendet und sollte im Normalfall 65*65px groß
sein.


\paragraph{Preise}
\label{\detokenize{hbe/products/fields/prices:preise}}\label{\detokenize{hbe/products/fields/prices::doc}}

\subparagraph{Preis fixieren}
\label{\detokenize{hbe/products/fields/prices:preis-fixieren}}
mit dieser Option können Sie bestimmen, dass
der eingegebene Preis für den Artikel fix ist. Die Gestaltung
im Designer hat dann keinen Einfluss auf den Preis mehr.
Diese Option eignet sich gut für Promotionartikel und
Sonderangebote.


\subparagraph{Flächenberechnung aktivieren}
\label{\detokenize{hbe/products/fields/prices:flachenberechnung-aktivieren}}
Schaltet die Flächenberechnung als Grundlage für die Kalkulation des
Preises ein (siehe Flächenberechnung).


\subparagraph{Preis}
\label{\detokenize{hbe/products/fields/prices:preis}}
Der Brutto\sphinxhyphen{}Preis des Produktes ohne jegliche
Gestaltung. Dieser Preis sollte exakt dem Preis in Ihrem Shop
entsprechen.


\subparagraph{Textpreis pro Buchstabe/pro Zeile}
\label{\detokenize{hbe/products/fields/prices:textpreis-pro-buchstabe-pro-zeile}}
der Preis der bei der Gestaltung mit Text zum tragen kommt. Je nach gewählter
Option, wird dieser für jeden Buchstaben oder pro Zeile
kalkuliert und auf den Artikelpreis aufgeschlagen.


\subparagraph{Textelement Preis}
\label{\detokenize{hbe/products/fields/prices:textelement-preis}}
zusätzlich zum Textpreis, können Sie hier
einen Aufpreis je Textelement eintragen der für jeden
Textblock addiert werden soll.


\subparagraph{Upload Preis}
\label{\detokenize{hbe/products/fields/prices:upload-preis}}
Preispauschale die aufgeschlagen werden soll
wenn der Benutzer beim Gestalten des Artikels ein eigenes
Bild verwendet.


\subparagraph{Verpackungseinheit}
\label{\detokenize{hbe/products/fields/prices:verpackungseinheit}}
Gibt an wie viele Artikel zu einer
Verpackungseinheit gehören. Dies bewirkt zum einen dass
bei der Eingabe der Anzahl im Designer nur in diesen
Schritten erhöht werden kann, zum Anderen bezieht sich die
Shirtnetwork Gebührenberechnung auf diese Einheit, statt
auf die Stückzahl.


\subparagraph{Steuersatz}
\label{\detokenize{hbe/products/fields/prices:steuersatz}}
der bei der Berechnung für diesen Artikel
berücksichtigt werden soll. Der Steuersatz sollte exakt dem in
Ihrem Shop System entsprechen um Fehler bei der
Preisberechnung zu vermeiden.


\section{Motive}
\label{\detokenize{hbe/logos/index:motive}}\label{\detokenize{hbe/logos/index::doc}}

\subsection{Motivkategorien}
\label{\detokenize{hbe/logos/categories:motivkategorien}}\label{\detokenize{hbe/logos/categories::doc}}

\subsection{Motive}
\label{\detokenize{hbe/logos/logos:motive}}\label{\detokenize{hbe/logos/logos::doc}}

\chapter{Weiterführende Links}
\label{\detokenize{index:weiterfuhrende-links}}\begin{itemize}
\item {} 
\DUrole{xref,std,std-ref}{genindex}

\item {} 
\DUrole{xref,std,std-ref}{search}

\end{itemize}



\renewcommand{\indexname}{Stichwortverzeichnis}
\printindex
\end{document}